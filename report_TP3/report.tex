\documentclass[a4paper,11pt]{article}
\usepackage[francais]{babel}
%% Prévu pour compiler avec lualatex
% \usepackage[utf8]{inputenc}
 \usepackage{fontspec}
\usepackage{libertine}
% \usepackage[T1]{fontenc}
\usepackage{graphicx}
\usepackage{fancyhdr}
\usepackage[top=2.5cm, bottom=2.5cm, left=2cm, right=2cm]{geometry}
\usepackage{listings}
\usepackage[utf8]{luainputenc}
\usepackage[hidelinks]{hyperref}
\usepackage{caption}

\lfoot{\bsc{Enseirb-Matmeca}}
\rfoot{Informatique --- 3\ieme{} année}

\pagestyle{fancy}
\begin{document}

\begin{titlepage}
  \begin{center}

    \begin{center}
      \includegraphics[width=4cm]{EM.jpg}
    \end{center}

    \vspace*{1cm}
        
    \rule{0.75\linewidth}{0.7mm}\\[0.4cm]
    {\Huge Rapport TP2 --- MPI\\[0.4cm]}
    \rule{0.75\linewidth}{0.7mm} \\[1.5cm]

    {\Large Bazire \bsc{Houssin}\\Sylvain \bsc{Vaglica}\\Stéphane \bsc{Castelli}\\[2cm]}
    {\Large Mardi 5 Novembre 2013}
  \end{center}
\end{titlepage}

\tableofcontents
\clearpage
\section{Introduction}

Le but de ce projet est de simuler les interactions gravitationnelles entre des corps placés dans un espace sans frottement, notament des corps de grande masse tels que les corps célestes. Lorsque ce nombre de corps est élevé, la puissance de calcul nécessaire pour déterminer toutes les attractions est importante, car chaque corps dans une certaine mesure influe sur tous les autres. Le point positif, c'est qu'il est trivial de paralléliser les calculs car ceux-ci sont tous indépendants. MPI est donc un moyen efficace pour réduire la durée d'exécution sur une machine parallèle.

La force de gravitation s'applique à l'infini dans toute les directions, selon la formule

\[
\| \vec{F}_{A\rightarrow B} \| = G \cdot \frac{M_A \cdot M_B}{\mathit{AB^3}} \cdot \mathit{\vec{AB}}
\]


Pour les tests, nous nous sommes inspirés du système solaire.

\section{Performances}



\section{Conclusion}

La puissance de MPI se ressent pleinement lorsque l'on a des calculs lourds sans dépendances entre eux. Alors, on augmentant le nombre de processus on peut diminuer la durée d'exécution. On peut alors constater un phénomène de rotation autour du Soleil, comme on pouvait s'y attendre. On a pu constater une amélioration des performances avec l'augmentation du nombre de processus, jusqu'au point où chaque corps est géré par un processus. Exécuté en séquentiel, sur un grand nombre de planètes, cela prendrait un temps considérable, alors que là c'est d'une extrème rapidité.



\end{document}
