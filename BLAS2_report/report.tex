\documentclass[a4paper,11pt]{article}
\usepackage[francais]{babel}
%% Prévu pour compiler avec lualatex
 \usepackage[utf8]{inputenc}
%\usepackage{fontspec}
%\usepackage{libertine}
\usepackage[T1]{fontenc}
\usepackage{graphicx}
\usepackage{fancyhdr}
\usepackage[top=2.5cm, bottom=2.5cm, left=2cm, right=2cm]{geometry}
\usepackage{listings}
%\usepackage[utf8]{luainputenc}
\usepackage[hidelinks]{hyperref}
\usepackage{caption}

\lfoot{\bsc{Enseirb-Matmeca}}
\rfoot{Informatique --- 3\ieme{} année}

\pagestyle{fancy}
\begin{document}

\begin{titlepage}
  \begin{center}

    \begin{center}
      \includegraphics[width=4cm]{EM.jpg}
    \end{center}

    \vspace*{1cm}
        
    \rule{0.75\linewidth}{0.7mm}\\[0.4cm]
    {\Huge Rapport TP5 --- BLAS\\[0.4cm]}
    \rule{0.75\linewidth}{0.7mm} \\[1.5cm]

    {\Large Bazire \bsc{Houssin}\\Sylvain \bsc{Vaglica}\\Stéphane \bsc{Castelli}\\[2cm]}
    {\Large Vendredi 17 Janvier 2014}
  \end{center}
\end{titlepage}

\tableofcontents
\clearpage
\section{Introduction}

Dans la continuation du TP précédent, nous avons tenté d'implémenter des fonctions faisant partie de BLAS, ceci dans le but de réaliser une factorisation LU d'une matrice de manière parallèle. Nous avons d'abord procédé à la création d'une version séquentielle scalaire, puis par bloc, pour enfin nous intéresser à la parallélisation grace à MPI. Chacune des versions faisant appel à la précédente, il était indispensable de procéder dans cet ordre. Il est intéressant de noter que plus l'on avançait dans les étapes, plus la difficulté augmentait. Nous détaillerons par la suite celles que nous avons rencontré.

\section{Version séquentielle}


\section{Version parallèle}

La version parallèle 

\section{Conclusion}

Réaliser une factorisation LU en utilisant MPI nécessite en premier lieu de bien comprendre comment réaliser une factorisation LU. En effet, comprendre ses mécanismes s'est effectué essentiellement par la lecture de code en Fortran, il a donc fallu assimiler la méthode --- ce qui a demandé un certain temps --- en conséquence de quoi la parallélisation, pourtant coeur du projet, n'a pas pu être aussi bien avancée qu'il l'aurait été souhaité. Toutefois le principe a pu être compris, et des progrès considérables auraient pu être accomplis en un laps de temps minime. En outre, cela nous a permis d'encore approfondir la mise en pratique de l'algèbre numérique couplée à la programmation informatique.

\end{document}
