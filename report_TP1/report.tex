\documentclass[a4paper,11pt]{article}
\usepackage[francais]{babel}
%% Prévu pour compiler avec lualatex
% \usepackage[utf8]{inputenc}
\usepackage{fontspec}
\usepackage{libertine}
% \usepackage[T1]{fontenc}
\usepackage{graphicx}
\usepackage{fancyhdr}
\usepackage[top=2.5cm, bottom=2.5cm, left=2cm, right=2cm]{geometry}
\usepackage{listings}
\usepackage[utf8]{luainputenc}
\usepackage[hidelinks]{hyperref}
\usepackage{caption}

\lfoot{\bsc{Enseirb-Matmeca}}
\rfoot{Informatique --- 3\ieme{} année}

\pagestyle{fancy}
\begin{document}

\begin{titlepage}
  \begin{center}

    \begin{center}
      \includegraphics[width=4cm]{EM.jpg}
    \end{center}

    \vspace*{1cm}
        
    \rule{0.75\linewidth}{0.7mm}\\[0.4cm]
    {\Huge Rapport TP1 --- PVM\\[0.4cm]}
    \rule{0.75\linewidth}{0.7mm} \\[1.5cm]

    {\Large Bazire Houssin\\Sylvain Vaglica\\Stéphane \bsc{Castelli}\\[2cm]}
    {\Large Mardi 29 Octobre 2013}
  \end{center}
\end{titlepage}

\tableofcontents
\clearpage
\section{Introduction}

PVM est une bibliothèque C et Fortran permettant la communication entre des processus sur une machine parallèle ou un cluster de machines ; elle utilise pour cela un démon lancé sur chaque machine qui se charge du routage des messages, du contrôle des processus et des problèmes pouvant survenir.Créée en 1989, PVM a au fil des décennies perdu en popularité au profit de MPI, mais reste un moyen efficace d'apporter une solution à des problèmes dont la résolution en séquentiel mettrait trop de temps. Au travers de ce projet, il a été réalisé un système distribué de craquage de mot de passe en force brute, sur le modèle maître / esclaves, avec un processus maître créant et distribuant des tâches, et des processus esclaves les exécutant. Un des objectifs principaux du projet, en plus de nous initier à la communication inter-processus et plus particulièrement à PVM, est d'analyser et optimiser la création et la répartition des tâches afin d'obtenir les meilleures performances possibles. Il sera d'abord explicité les choix qui ont été effectués, puis ils seront analysés.

\section{Le maître}

\section{Les esclaves}

\section{La répartition des tâches}

\section{Performances}

\section{Conclusion}


\end{document}