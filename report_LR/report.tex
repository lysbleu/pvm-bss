\documentclass[a4paper,11pt]{article}
\usepackage[francais]{babel}
%% Prévu pour compiler avec lualatex
 \usepackage[utf8]{inputenc}
%\usepackage{fontspec}
%\usepackage{libertine}
\usepackage[T1]{fontenc}
\usepackage{graphicx}
\usepackage{fancyhdr}
\usepackage[top=2.5cm, bottom=2.5cm, left=2cm, right=2cm]{geometry}
\usepackage{listings}
%\usepackage[utf8]{luainputenc}
\usepackage[hidelinks]{hyperref}
\usepackage{caption}

\lfoot{\bsc{Enseirb-Matmeca}}
\rfoot{Informatique --- 3\ieme{} année}

\pagestyle{fancy}
\begin{document}

\begin{titlepage}
  \begin{center}

    \begin{center}
      \includegraphics[width=4cm]{EM.jpg}
    \end{center}

    \vspace*{1cm}
        
    \rule{0.75\linewidth}{0.7mm}\\[0.4cm]
    {\Huge Rapport TP6 --- Lancer de rayons\\[0.4cm]}
    \rule{0.75\linewidth}{0.7mm} \\[1.5cm]

    {\Large Bazire \bsc{Houssin}\\Sylvain \bsc{Vaglica}\\Stéphane \bsc{Castelli}\\[2cm]}
    {\Large Dimanche 2 Février 2014}
  \end{center}
\end{titlepage}

\tableofcontents
\clearpage
\section{Introduction}

Afin de représenter une scène, potentiellement dynamique, sur un ordinateur, une technique courante est le lancer de rayons (\emph{ray tracing}). Celle-ci se base sur le principe de retour inverse de la lumière, en effet au lieu de calculer les trajectoires des rayons de lumière pour chacune des sources, ce qui représente un gaspillage de ressources car la plupart de ceux-ci n'atteignent pas l'observateur, on calcule la trajectoire de rayons de lumière qui partiraient de l'observateur pour aller vers les surfaces environnantes. Ainsi, on ne calcule que ce qui est observable. Cette technique a ceci de particulier qu'elle est fortement parallélisable, car chacun des rayons est indépendant des autres. En partant de la version séquentielle de l'algorithme, on implémentera une version parallèle. Ensuite, on mettera en pratique la technique du \emph{vol de travail}.

\section{Version parallèle}

Paralléliser un lancer de rayons consiste essentiellement à distribuer les différentes tâches de manière homogène entre les processus. En effet, étant donné qu'il n'y  a pas de dépendances entres les données, chaque processus exécute les calculs sur une fraction des rayons, de manière séquentielle, sans se préoccuper du restant de l'exécution. La version séquentielle étant fournie, la version parallèle se greffe sur celle-ci sans modifier outre mesure son code.

De manière pratique, un rayon correspond à un pixel, et les pixels sont regroupés par groupe de $8 \times 8$ ou $16 \times 16$ contigus sur l'image pour créer une tâche. Ces groupes de pixels seront appelés \emph{carreaux}. 

Afin d'obtenir des performances optimales, la répartition des tâches doit se faire sans communications. Autrement dit, chaque processus doit pouvoir déterminer par lui même quelles sont ses tâches, ceci sans créer de conflits avec les autres processus. Une solution pourrait être que chaque processus se charge des $N$ premiers carreaux ($N = \left[\frac{\#\mathrm{rayons}}{\#\mathrm{processus}}\right]$)


\section{Vol de travail}

Le vol de travail consiste, lorsqu'un processus n'a plus de tâches à accomplir, à en demander aux autres processus. Un message est envoyé au processus suivant sur l'anneau, et l'on attend la réponse. Lorsqu'un processus reçoit une demande de travail, il regarde s'il lui en reste sur sa file de tâche. Si oui, alors une réponse positive est envoyée directement au processus qui l'a demandé. Si non, alors la demande est transférée au processus suivant.

Le processus oisif recevra forcément une réponse, que ça soit du travail à accomplir ou sa propre demande. Si l'on se trouve dans ce dernier cas, alors cela signifie que l'exécution de tous les calculs est sur le point de se terminer, on envoie sur l'anneau un message de terminaison. Lorsqu'un message de terminaison est reçu, cela signifie qu'un autre processus a cherché du travail et qu'il n'en a pas trouvé. Par conséquent, lorsque le processus doit chercher du travail, il ne le fait pas et se termine à la place.

Au niveau de l'implémentation, 


\section{Conclusion}
Le premier objectif de ce TP était avant tout de parallèliser le code de calcul du lancer de rayons en mettant en place une nouvelle politique de partage des données. En effet l'image traitée pouvant être très irrégulière, il était essentiel d'assigner les carreaux à chaque processus de manière homogène sur toute l'étendue de l'image. Malgrès celà, la question de l'équilibrage des charges n'était pas encore totalement résolue, car même si tous les processus ont initialement autant de données initiales à traiter, réparties sur toute l'image traitée, il est toujours possible que certains finissent largement avant les autres et restent inactifs jusqu'à la terminaison de tous les processus.  Il a donc été décidé d'utiliser une structure en anneau pour mettre en place un politique de vol de travail, où chaque processus qui aurait consommé toutes ses données enverrai une requette sur l'anneau pour en obtenir de nouvelles, prises dans les données restantes d'un autre processus. Une nouvelle problématique apparaît alors, et a été une des difficultés importantes abordées dans ce TP: la question de la détection de la terminaison de l'algorithme, sur l'anneau, lorsque tous les noeuds ont terminé leur travail.

\end{document}
