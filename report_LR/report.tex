\documentclass[a4paper,11pt]{article}
\usepackage[francais]{babel}
%% Prévu pour compiler avec lualatex
 \usepackage[utf8]{inputenc}
%\usepackage{fontspec}
%\usepackage{libertine}
\usepackage[T1]{fontenc}
\usepackage{graphicx}
\usepackage{fancyhdr}
\usepackage[top=2.5cm, bottom=2.5cm, left=2cm, right=2cm]{geometry}
\usepackage{listings}
%\usepackage[utf8]{luainputenc}
\usepackage[hidelinks]{hyperref}
\usepackage{caption}

\lfoot{\bsc{Enseirb-Matmeca}}
\rfoot{Informatique --- 3\ieme{} année}

\pagestyle{fancy}
\begin{document}

\begin{titlepage}
  \begin{center}

    \begin{center}
      \includegraphics[width=4cm]{EM.jpg}
    \end{center}

    \vspace*{1cm}
        
    \rule{0.75\linewidth}{0.7mm}\\[0.4cm]
    {\Huge Rapport TP6 --- Lancer de rayons\\[0.4cm]}
    \rule{0.75\linewidth}{0.7mm} \\[1.5cm]

    {\Large Bazire \bsc{Houssin}\\Sylvain \bsc{Vaglica}\\Stéphane \bsc{Castelli}\\[2cm]}
    {\Large Dimanche 2 Février 2014}
  \end{center}
\end{titlepage}

\tableofcontents
\clearpage
\section{Introduction}

Afin de représenter une scène, potentiellement dynamique, sur un ordinateur, une technique courante est le lancer de rayons (\emph{ray tracing}). Celle-ci se base sur le principe de retour inverse de la lumière, en effet au lieu de calculer les trajectoires des rayons de lumière pour chacune des sources, ce qui représente un gaspillage de ressources car la plupart de ceux-ci n'atteignent pas l'observateur, on calcule la trajectoire de rayons de lumière qui partiraient de l'observateur pour aller vers les surfaces environnantes. Ainsi, on ne calcule que ce qui est observable. Cette technique a ceci de particulier qu'elle est fortement parallélisable, car chacun des rayons est indépendant des autres. En partant de la version séquentielle de l'algorithme, on implémentera une version parallèle. Ensuite, on mettera en pratique la technique du \emph{vol de travail}.

\section{Version parallèle}


\section{Vol de travail}


\section{Conclusion}


\end{document}
